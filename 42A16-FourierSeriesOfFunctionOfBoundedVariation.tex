\documentclass[12pt]{article}
\usepackage{pmmeta}
\pmcanonicalname{FourierSeriesOfFunctionOfBoundedVariation}
\pmcreated{2013-03-22 17:58:00}
\pmmodified{2013-03-22 17:58:00}
\pmowner{pahio}{2872}
\pmmodifier{pahio}{2872}
\pmtitle{Fourier series of function of bounded variation}
\pmrecord{7}{40472}
\pmprivacy{1}
\pmauthor{pahio}{2872}
\pmtype{Theorem}
\pmcomment{trigger rebuild}
\pmclassification{msc}{42A16}
\pmclassification{msc}{42A20}
\pmclassification{msc}{26A45}
\pmrelated{DirichletConditions}
\pmrelated{FourierCoefficients}

\endmetadata

% this is the default PlanetMath preamble.  as your knowledge
% of TeX increases, you will probably want to edit this, but
% it should be fine as is for beginners.

% almost certainly you want these
\usepackage{amssymb}
\usepackage{amsmath}
\usepackage{amsfonts}

% used for TeXing text within eps files
%\usepackage{psfrag}
% need this for including graphics (\includegraphics)
%\usepackage{graphicx}
% for neatly defining theorems and propositions
 \usepackage{amsthm}
% making logically defined graphics
%%%\usepackage{xypic}

% there are many more packages, add them here as you need them

% define commands here

\theoremstyle{definition}
\newtheorem*{thmplain}{Theorem}

\begin{document}
\PMlinkescapeword{expansion}

If the real function $f$ is of bounded variation on the interval \,$[-\pi,\,+\pi]$,\, then its Fourier series expansion
\begin{align}
\frac{a_0}{2}+\sum_{n=1}^\infty(a_n\cos{nx}+b_n\sin{nx})
\end{align}
with the \PMlinkname{coefficients}{FourierCoefficients}
\begin{align}
\begin{cases}
a_n &= \frac{1}{\pi}\int_{-\pi}^{\pi} f(x)\cos{nx}\,dx\\
b_n &= \frac{1}{\pi}\int_{-\pi}^{\pi} f(x)\sin{nx}\,dx
\end{cases}
\end{align}
converges at every point of the interval. The sum of the series is at the interior points $x$ equal to the arithmetic mean of the \PMlinkname{left-sided}{OneSidedLimit} and the right-sided limit of $f$ at $x$ and at the end-points of the interval equal to\,
$\displaystyle\frac{1}{2}\left(\lim_{x\to-\pi+}f(x)+\lim_{x\to+\pi-}f(x)\right)$.\\

\textbf{Remark 1.}\, Because of the periodicity of the terms of the terms, the expansion (1) converges for all real values of $x$ and it represents a periodic function with the period $2\pi$.\\

\textbf{Remark 2.}\, If the function $f$ is of bounded variation, instead of\, $[-\pi,\,+\pi]$,\, on the interval 
\,$[-p,\,+p]$\, the equations (1) and (2) may be converted via the change of variable 
\,$\displaystyle x := \frac{pt}{\pi}$\, to
\begin{align}
\frac{a_0}{2}+\sum_{n=1}^\infty(a_n\cos\frac{n\pi t}{p}+b_n\sin\frac{n\pi t}{p})
\end{align}
and
\begin{align}
\begin{cases}
a_n &= \frac{1}{p}\int_{-p}^p f(t)\cos\frac{n\pi t}{p}\,dt\\
b_n &= \frac{1}{p}\int_{-p}^p f(t)\sin\frac{n\pi t}{p}\,dt.
\end{cases}
\end{align}
Correspondingly, the sum of (3) at the points of\, $[-p,\,+p]$\, is expressed via the left-sided and righr-sided limits of $f(t)$.



%%%%%
%%%%%
\end{document}
