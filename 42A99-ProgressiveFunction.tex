\documentclass[12pt]{article}
\usepackage{pmmeta}
\pmcanonicalname{ProgressiveFunction}
\pmcreated{2013-03-22 14:28:20}
\pmmodified{2013-03-22 14:28:20}
\pmowner{swiftset}{1337}
\pmmodifier{swiftset}{1337}
\pmtitle{progressive function}
\pmrecord{4}{35993}
\pmprivacy{1}
\pmauthor{swiftset}{1337}
\pmtype{Definition}
\pmcomment{trigger rebuild}
\pmclassification{msc}{42A99}
\pmrelated{FourierTransform}
\pmdefines{regressive function}

% this is the default PlanetMath preamble.  as your knowledge
% of TeX increases, you will probably want to edit this, but
% it should be fine as is for beginners.

% almost certainly you want these
\usepackage{amssymb}
\usepackage{amsmath}
\usepackage{amsfonts}

% used for TeXing text within eps files
%\usepackage{psfrag}
% need this for including graphics (\includegraphics)
%\usepackage{graphicx}
% for neatly defining theorems and propositions
%\usepackage{amsthm}
% making logically defined graphics
%%%\usepackage{xypic}

% there are many more packages, add them here as you need them

% define commands here
\begin{document}
\newcommand{\supp}{\ensuremath{\mathop{\rm supp}}}
\newcommand{\R}{\ensuremath{\mathbb R}}

\paragraph{Definition}
A function $f \in L^2(\R)$ is called \emph{progressive} iff its Fourier transform is supported by positive frequencies only:
$$\supp \hat{f} \subseteq \R_+.$$
It is called \emph{regressive} iff the time reversed function $f(-t)$ is progressive, or equivalently, if 
$$\supp \hat{f} \subseteq \R_-.$$
%%%%%
%%%%%
\end{document}
