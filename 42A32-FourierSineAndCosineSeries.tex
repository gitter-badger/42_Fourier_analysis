\documentclass[12pt]{article}
\usepackage{pmmeta}
\pmcanonicalname{FourierSineAndCosineSeries}
\pmcreated{2013-03-22 15:42:20}
\pmmodified{2013-03-22 15:42:20}
\pmowner{pahio}{2872}
\pmmodifier{pahio}{2872}
\pmtitle{Fourier sine and cosine series}
\pmrecord{26}{37650}
\pmprivacy{1}
\pmauthor{pahio}{2872}
\pmtype{Topic}
\pmcomment{trigger rebuild}
\pmclassification{msc}{42A32}
\pmclassification{msc}{42A20}
\pmclassification{msc}{42A16}
\pmclassification{msc}{26A06}
%\pmkeywords{odd function}
%\pmkeywords{even function}
\pmrelated{SubstitutionNotation}
\pmrelated{IntegralsOfEvenAndOddFunctions}
\pmrelated{CosineAtMultiplesOfStraightAngle}
\pmrelated{ExampleOfFourierSeries}
\pmrelated{DoubleSeries}
\pmrelated{UniquenessOfFourierExpansion}
\pmrelated{DeterminationOfFourierCoefficients}
\pmrelated{TwoDimensionalFourierTransforms}
\pmdefines{Fourier sine series}
\pmdefines{Fourier cosine series}
\pmdefines{sine series}
\pmdefines{cosine series}
\pmdefines{half-interval}
\pmdefines{Fourier double sine series}
\pmdefines{Fourier double cosine series}

% this is the default PlanetMath preamble.  as your knowledge
% of TeX increases, you will probably want to edit this, but
% it should be fine as is for beginners.

% almost certainly you want these
\usepackage{amssymb}
\usepackage{amsmath}
\usepackage{amsfonts}

% used for TeXing text within eps files
%\usepackage{psfrag}
% need this for including graphics (\includegraphics)
%\usepackage{graphicx}
% for neatly defining theorems and propositions
 \usepackage{amsthm}
% making logically defined graphics
%%%\usepackage{xypic}

% there are many more packages, add them here as you need them

% define commands here
\newcommand{\sijoitus}[2]%
{\operatornamewithlimits{\Big/}_{\!\!\!#1}^{\,#2}}


\theoremstyle{definition}
\newtheorem*{thmplain}{Theorem}
\begin{document}
One sees from the formulae 
\begin{align*}
a_n &= \frac{1}{\pi}\int_{-\pi}^{\pi} f(x)\cos{nx}\,dx,\\
b_n &= \frac{1}{\pi}\int_{-\pi}^{\pi} f(x)\sin{nx}\,dx
\end{align*}
of the coefficients $a_n$ and $b_n$ for the Fourier series \PMlinkescapetext{expansion}
$$f(x) = \frac{a_0}{2}+\sum_{n=1}^\infty(a_n\cos{nx}+b_n\sin{nx})$$
of the Riemann integrable real function $f$ on the interval\, $[-\pi,\,\pi]$,\, that 
\begin{itemize}
\item $\displaystyle a_n = \frac{2}{\pi}\int_0^\pi\!f(x)\cos{nx}\,dx$,\,\, 
$b_n = 0$\, $\forall n$\, if $f$ is an even function;
\item $\displaystyle b_n = \frac{2}{\pi}\int_0^\pi\!f(x)\sin{nx}\,dx$,\,\, 
$a_n = 0$\, $\forall n$\, if $f$ is an odd function.
\end{itemize}
Thus the Fourier series of an even function \PMlinkescapetext{contains} mere cosine \PMlinkescapetext{terms} and of an odd function mere sine \PMlinkescapetext{terms}.\, This concerns the whole interval\, 
$[-\pi,\,\pi]$.\, So e.g. one has on this interval
$$x \,\equiv\, 
 2\!\left(\frac{\sin{x}}{1}\!-\!\frac{\sin{2x}}{2}\!+\!\frac{\sin{3x}}{3}\!
-+\cdots\right).$$\\

\textbf{Remark 1}.\, On the {\em half-interval}\, $[0,\,\pi]$\, one can in any case expand each Riemann integrable function $f$ both to a cosine series and to a sine series, irrespective of how it is defined for the negative half-interval or is it defined here at all.\\

\textbf{Remark 2.}\, On an interval\, $[-p,\,p]$,\, instead of\, 
$[-\pi,\,\pi]$,\, the Fourier coefficients of the series
$$f(x) = 
\frac{a_0}{2}+\sum_{n=1}^\infty\left(a_n\cos\frac{n\pi x}{p}
+b_n\sin\frac{n\pi x}{p}\right)$$
have the expressions
\begin{itemize}
\item $\displaystyle a_n = 
 \frac{2}{p}\int_0^p\!f(x)\cos\frac{n\pi x}{p}\,dx$,\,\, 
$b_n = 0$\, $\forall n$\, if $f$ is an even function;
\item $\displaystyle b_n =
 \frac{2}{p}\int_0^p\!f(x)\sin\frac{n\pi x}{p}\,dx$,\,\,
$a_n = 0$\, $\forall n$\, if $f$ is an odd function.
\end{itemize}


\textbf{Example.}\, Expand the \PMlinkname{identity function}{IdentityMap}\, $x\mapsto x$\, to a Fourier cosine series on\, $[0,\,\pi]$.

This odd function may be replaced with the even function\, $f: x\mapsto |x|$.\, Then we get
 $$a_0 = \frac{2}{\pi}\int_0^\pi x\,dx = \pi$$
and, integrating by parts,
 $$a_n = \frac{2}{\pi}\int_0^\pi\!x\cos{nx}\,dx 
= \frac{2}{\pi}\left[\sijoitus{0}{\quad\pi}\!x\frac{\sin{nx}}{n}
                   -\int_0^\pi \frac{\sin{nx}}{n}\,dx\right] 
= \frac{2}{\pi}\sijoitus{0}{\quad\pi}\!\frac{\cos{nx}}{n^2}
= \frac{2}{\pi n^2}((-1)^n\!-\!1));$$
this equals to $\displaystyle-\frac{4}{\pi n^2}$ if $n$ is an odd integer, but vanishes for each even $n$.\, Thus we obtain on the interval\, $[0,\,\pi]$\, the cosine series
$$x \,\equiv\, 
\frac{\pi}{2}\!-\!\frac{4}{\pi}\!\left(\frac{\cos{x}}{1^2}\!+\!\frac{\cos{3x}}{3^2}
          \!+\!\frac{\cos{5x}}{5^2}\!+\cdots\right).$$
Chosing here\, $x := 0$\, one gets the result
$$\frac{\pi^2}{8} \;=\; 1+\frac{1}{3^2}+\frac{1}{5^2}+\ldots$$
(cf. the entry on \PMlinkid{Dirichlet eta function at 2}{11010}).\\


\textbf{Fourier double series.}\, The Fourier sine and cosine series introduced in Remark 1 on the half-interval\, $[0,\,\pi]$\, for a function of one real variable may be generalized for e.g. functions of two real variables on a rectangle\, $\{(x,\,y)\in \mathbb{R}^2\,\vdots\,\, 0\le x \le a,\,0\le y \le b\}$:
\begin{align}
f(x,\,y) = \sum_{m=1}^\infty\sum_{n=1}^\infty c_{mn}\sin\frac{m\pi x}{a}
\sin\frac{n\pi y}{b},
\end{align}
\begin{align}
f(x,\,y) = \frac{d_{00}}{4}+\frac{1}{2}\sum_{l=1}^\infty \left(d_{l0}\cos\frac{l\pi x}{a}+d_{0l}\cos\frac{l\pi y}{b}\right)+
\sum_{m=1}^\infty\sum_{n=1}^\infty d_{mn}\cos\frac{m\pi x}{a}
\cos\frac{n\pi y}{b}
\end{align}
The coefficients of the {\em Fourier double sine series} (1) are got by the double integral
$$c_{mn} = \frac{4}{ab}
\int_0^a\int_0^b f(x,\,y)\,\sin\frac{m\pi x}{a}\sin\frac{n\pi y}{b}\,dx\,dy$$
where\, $m = 1,\,2,\,3,\,\ldots$\, and\, $n = 1,\,2,\,3,\,\ldots$\, The coefficients of the {\em Fourier double cosine series} (2) are correspondingly
$$d_{mn} = \frac{4}{ab}
\int_0^a\int_0^b f(x,\,y)\,\cos\frac{m\pi x}{a}\cos\frac{n\pi y}{b}\,dx\,dy$$
where\, $m = 0,\,1,\,2,\,\ldots$\, and\, $n = 0,\,1,\,2,\,\ldots$\\

\textbf{Note.}\, One can use in the double series of (1) and (2) also the {\em diagonal summing}, e.g. for the double sine series as follows:\\
$c_{11}\sin\!\frac{\pi x}{a}\sin\!\frac{\pi y}{b}\!+\!
\left(c_{12}\sin\!\frac{\pi x}{a}\sin\!\frac{2\pi y}{b}\!+\!
      c_{21}\sin\!\frac{2\pi x}{a}\sin\!\frac{\pi y}{b}\right)\!+\!
\left(c_{13}\sin\!\frac{\pi x}{a}\sin\!\frac{3\pi y}{b}\!+\!
      c_{22}\sin\!\frac{2\pi x}{a}\sin\!\frac{2\pi y}{b}\!+\!
      c_{31}\sin\!\frac{3\pi x}{a}\sin\!\frac{\pi y}{b}\right)\!+\ldots$

\begin{thebibliography}{9}
\bibitem{K.V.}{\sc K. V\"ais\"al\"a:} {\em Matematiikka IV}.\, Hand-out Nr. 141.\quad Teknillisen korkeakoulun ylioppilaskunta, Otaniemi, Finland (1967).
\end{thebibliography}
%%%%%
%%%%%
\end{document}
