\documentclass[12pt]{article}
\usepackage{pmmeta}
\pmcanonicalname{EquivalentConditionForTheTranslatesOfAnL2FunctionToFormARieszSequenceAn}
\pmcreated{2013-03-22 15:20:07}
\pmmodified{2013-03-22 15:20:07}
\pmowner{Gorkem}{3644}
\pmmodifier{Gorkem}{3644}
\pmtitle{equivalent condition for the translates of an $L_2$ function to form a Riesz sequence, an}
\pmrecord{15}{37152}
\pmprivacy{1}
\pmauthor{Gorkem}{3644}
\pmtype{Theorem}
\pmcomment{trigger rebuild}
\pmclassification{msc}{42C99}
%\pmkeywords{Riesz basis}
%\pmkeywords{wavelet theory}
%\pmkeywords{fourier analysis}
%\pmkeywords{sampling theory}
\pmrelated{RieszSequence}

\endmetadata

% this is the default PlanetMath preamble.  as your knowledge
% of TeX increases, you will probably want to edit this, but
% it should be fine as is for beginners.

% almost certainly you want these
\usepackage{amssymb}
\usepackage{amsmath}
\usepackage{amsfonts}

% used for TeXing text within eps files
%\usepackage{psfrag}
% need this for including graphics (\includegraphics)
%\usepackage{graphicx}
% for neatly defining theorems and propositions
%\usepackage{amsthm}
% making logically defined graphics
%%%\usepackage{xypic}

% there are many more packages, add them here as you need them

\renewcommand{\labelenumi}{(\roman{enumi})}
\newtheorem{theorem}{Theorem}
\begin{document}
\begin{theorem}
Let $\phi \in L_2(\mathbb{R})$, $\phi_k(x) = \phi(x-k)$ and
$\hat{\phi}$ be the Fourier transform of $\phi$. Let $A$ and $B$
be positive constants. Then the following are equivalent:
\begin{enumerate}
\item $\forall c(k) \in l_2,\ \   A\left\|c\right\|^2_{l_2} \leq
\left\|\sum_{k\in \mathbb{Z}}c(k)\phi_k\right\|^2\leq
B\left\|c\right\|^2_{l_2}$ \item $
A\leq\sum_{k\in\mathbb{Z}}\left|\hat{\phi}(\omega + 2\pi
k)\right|^2\leq B$
\end{enumerate}
\end{theorem}

The first of the above conditions is the definition for
$\{\phi_k\}_{k\in\mathbb{Z}}$ to form a Riesz sequence. 
%%%%%
%%%%%
\end{document}
