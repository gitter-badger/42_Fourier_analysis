\documentclass[12pt]{article}
\usepackage{pmmeta}
\pmcanonicalname{GeneralizedRiemannLebesgueLemma}
\pmcreated{2013-03-22 17:06:03}
\pmmodified{2013-03-22 17:06:03}
\pmowner{fernsanz}{8869}
\pmmodifier{fernsanz}{8869}
\pmtitle{generalized Riemann-Lebesgue lemma}
\pmrecord{13}{39398}
\pmprivacy{1}
\pmauthor{fernsanz}{8869}
\pmtype{Theorem}
\pmcomment{trigger rebuild}
\pmclassification{msc}{42A16}
%\pmkeywords{Riemann}
%\pmkeywords{Lebesgue}
%\pmkeywords{Riemann-Lebesgue}
%\pmkeywords{fourier coefficientes}
%\pmkeywords{trigonometric series}
\pmrelated{RiemannLebesgueLemma}
\pmrelated{FourierCoefficients}
\pmrelated{Integral2}

% this is the default PlanetMath preamble.  as your knowledge
% of TeX increases, you will probably want to edit this, but
% it should be fine as is for beginners.

% almost certainly you want these
\usepackage{amssymb}
\usepackage{amsmath}
\usepackage{amsfonts}
\usepackage{amsthm}

% there are many more packages, add them here as you need them

% define commands here
\newtheorem{lem}{Lemma}
\newcommand{\norm}[1]{\left\Vert#1\right\Vert}
\newcommand{\abs}[1]{\left\vert#1\right\vert}
\newcommand{\Real}{\mathbb R}
\newcommand{\Complex}{\mathbb C}
\newcommand{\To}{\longrightarrow}
\begin{document}
\title{Generalized Riemann-Lebesgue lemma}%
\author{Fernando Sanz Gamiz}%


\begin{lem} Let $h \colon \Real \to \Complex$ be a bounded measurable function.
If $h$ satisfies the \emph{averaging condition} $$\lim_{c \to
+\infty} \frac{1}{c} \int_0^c h(t) \,d t=0$$ then $$\lim_{\omega \to
\infty} \int_a^b f(t)h(\omega t) \,d t=0$$ with $ -\infty <
\! a < b < \! +\infty$ for any $f \in L^1[a,b]$

\end{lem}

\begin{proof}
Obviously we only need to prove the lemma when both $h$ and $f$ are
real and $0=a<b<\infty$.

\smallskip

\noindent Let $\mathbf 1_{[a,b]}$ be the indicator function of the
interval $[a,b]$. Then
$$\lim_{\omega \to \infty} \int_0^b \mathbf 1_{[a,b]} h(\omega t) \,d t= \lim_{\omega \to \infty} \frac{1}{\omega} \int_0^{\omega b} h(t) \,d
t=0$$ by the hypothesis. Hence, the lemma is valid for indicators,
therefore for step functions.

\noindent Now let $C$ be a bound for $h$ and choose $\epsilon$ $>0$.
As step functions are dense in $L^1$, we can find, for any $f\in
L^1[a,b]$, a step function $g$ such that $\norm{f-g}_1<\epsilon$,
therefore

\begin{eqnarray*}
\lim_{\omega \to \infty} \abs{\int_a^b f(t)h(\omega t)\,d t}  &
\leqslant & \lim_{\omega \to \infty} \int_a^b \abs{f(t)-g(t)}
\abs{h(\omega t)} \,d t + \lim_{\omega \to \infty} \abs{\int_a^b
g(t)h(\omega t) \,d t}
\\ & \leqslant & \lim_{\omega \to \infty} C\norm{f-g}_1 < C\epsilon
\end{eqnarray*}

\noindent because $\lim_{\omega \to \infty} \abs{\int_a^b
g(t)h(\omega t) \,d t}=0$ by what we have proved for step
functions. Since $\epsilon$ is arbitrary, we are done.

\end{proof}
%%%%%
%%%%%
\end{document}
