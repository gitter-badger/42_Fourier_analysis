\documentclass[12pt]{article}
\usepackage{pmmeta}
\pmcanonicalname{CountableBasis}
\pmcreated{2013-03-22 12:10:37}
\pmmodified{2013-03-22 12:10:37}
\pmowner{djao}{24}
\pmmodifier{djao}{24}
\pmtitle{countable basis}
\pmrecord{8}{31434}
\pmprivacy{1}
\pmauthor{djao}{24}
\pmtype{Definition}
\pmcomment{trigger rebuild}
\pmclassification{msc}{42-00}
\pmclassification{msc}{15A03}
\pmsynonym{Schauder basis}{CountableBasis}

\endmetadata

% this is the default PlanetMath preamble.  as your knowledge
% of TeX increases, you will probably want to edit this, but
% it should be fine as is for beginners.

% almost certainly you want these
\usepackage{amssymb}
\usepackage{amsmath}
\usepackage{amsfonts}

% used for TeXing text within eps files
%\usepackage{psfrag}
% need this for including graphics (\includegraphics)
%\usepackage{graphicx}
% for neatly defining theorems and propositions
%\usepackage{amsthm}
% making logically defined graphics
%%%%\usepackage{xypic} 

% there are many more packages, add them here as you need them

% define commands here
\begin{document}
A {\em countable basis} $\beta$ of a vector space $V$ over a field $F$ is a countable subset $\beta \subset V$ with the property that every element $v \in V$ can be written as an infinite series
$$
v = \sum_{x \in \beta} a_x x
$$
in exactly one way (where $a_x \in F$). We are implicitly assuming, without further comment, that the vector space $V$ has been given a topological structure or normed structure in which the above infinite sum is absolutely convergent (so that it converges to $v$ regardless of the order in which the terms are summed).

The archetypical example of a countable basis is the Fourier series of a function: every continuous real-valued periodic function $f$ on the unit circle $S^1 = \mathbb{R}/2\pi$ can be written as a Fourier series
$$
f(x) = \sum_{n=0}^\infty a_n \cos(n x) + \sum_{n=1}^\infty b_n \sin(n x)
$$
in exactly one way.

Note: A countable basis is a countable set, but it is not usually a basis.
%%%%%
%%%%%
%%%%%
\end{document}
