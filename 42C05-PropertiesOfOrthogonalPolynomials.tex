\documentclass[12pt]{article}
\usepackage{pmmeta}
\pmcanonicalname{PropertiesOfOrthogonalPolynomials}
\pmcreated{2013-03-22 19:05:34}
\pmmodified{2013-03-22 19:05:34}
\pmowner{pahio}{2872}
\pmmodifier{pahio}{2872}
\pmtitle{properties of orthogonal polynomials}
\pmrecord{12}{41983}
\pmprivacy{1}
\pmauthor{pahio}{2872}
\pmtype{Topic}
\pmcomment{trigger rebuild}
\pmclassification{msc}{42C05}
\pmclassification{msc}{33D45}
\pmrelated{HilbertSpace}
\pmrelated{TopicsOnPolynomials}
\pmrelated{IndexOfSpecialFunctions}
\pmrelated{OrthogonalityOfLaguerrePolynomials}
\pmrelated{OrthogonalityOfChebyshevPolynomials}
\pmdefines{Rodrigues formula}

% this is the default PlanetMath preamble.  as your knowledge
% of TeX increases, you will probably want to edit this, but
% it should be fine as is for beginners.

% almost certainly you want these
\usepackage{amssymb}
\usepackage{amsmath}
\usepackage{amsfonts}

% used for TeXing text within eps files
%\usepackage{psfrag}
% need this for including graphics (\includegraphics)
%\usepackage{graphicx}
% for neatly defining theorems and propositions
 \usepackage{amsthm}
% making logically defined graphics
%%%\usepackage{xypic}

% there are many more packages, add them here as you need them

% define commands here

\theoremstyle{definition}
\newtheorem*{thmplain}{Theorem}

\begin{document}
A countable \emph{system of orthogonal polynomials}
\begin{align}
p_0(x),\,p_1(x),\,p_2(x),\,\ldots
\end{align}
on an interval \,$[a,\,b]$,\, where a inner product of two functions 
$$(f,\,g) \;:=\; \int_a^b\!f(x)g(x)W(x)\,dx$$
is defined with respect to a weighting function $W(x)$, satisfies the \PMlinkname{orthogonality condition}{OrthogonalVectors}
$$(p_m,\,p_n) \;=\; 0 \quad \mbox{always when}\quad m \neq n.$$
One also requires that
$$\deg\left(p_n(x)\right) \;=\; n \quad \mbox{for all } n.$$\\


Such a system (1) may be used as basis for the vector space of functions defined on\, $[a,\,b]$, i.e. certain such functions $f$ may be expanded as a \PMlinkname{series}{FunctionSeries}
$$f(x) \;=\; c_0p_0(x)+c_1p_1(x)+c_2p_2(x)+\ldots$$
where the coefficients $c_n$ have the expression
$$c_n \;=\; \int_a^b\!f(x)p_n(x)W(x)\,dx.$$\\

\textbf{Other properties}

\begin{itemize}

\item The basis property of the system (1) comprises that any polynomial $P(x)$ of degree $n$ can be uniquely expressed as a finite linear combination$\!$
$$P(x) \;=\; c_0p_0(x)+c_1p_1(x)+\ldots+c_np_n(x).$$

\item Every member $p_n(x)$ of (1) is orthogonal to any polynomial $P(x)$ of degree less than $n$.

\item There is a recurrence relation
$$p_{n+1}(x) \;=\; (a_nx\!+\!b_n)p_n(x)+c_np_{n-1}(x)$$
enabling to determine a \PMlinkescapetext{member of (1) in terms of the two preceding members}.

\item The zeros of $p_n(x)$ are all real and belong to the open interval \,$(a,\,b)$;\, between two of those zeros there are always zeros of $p_{n+1}(x)$.

\item The Sturm--Liouville differential equation
\begin{align}
Q(x)p''+L(x)p'+\lambda p \;=\; 0,
\end{align}
where $Q(x)$ is a polynomial of at most degree 2 and $L(x)$ a linear polynomial, gives under certain conditions, as \PMlinkid{solutions}{8719} $p$ a system of orthogonal polynomials $p_0,\,p_1,\,\ldots$\, corresponding suitable values (eigenvalues) $\lambda_0,\,\lambda_1,\,\ldots$\, of the parametre $\lambda$.\, Those \PMlinkescapetext{solutions} satisfy the Rodrigues formula
$$p_n(x) \;=\; \frac{k_n}{W(x)}\frac{d^n}{dx^n}\left(W(x)[Q(x)]^n\right),$$
where $k_n$ is a constant and
$$W(x) \;:=\; \frac{1}{Q(x)}e^{\int\frac{L(x)}{Q(x)}dx}.$$
The classical \PMlinkname{Chebyshev}{ChebyshevPolynomial}, \PMlinkname{Hermite}{HermitePolynomials}, \PMlinkname{Laguerre}{LaguerrePolynomial}, and Legendre polynomials all satisfy an equation (2).

\end{itemize}

[Not ready . . .]
%%%%%
%%%%%
\end{document}
