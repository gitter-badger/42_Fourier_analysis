\documentclass[12pt]{article}
\usepackage{pmmeta}
\pmcanonicalname{MinimalityPropertyOfFourierCoefficients}
\pmcreated{2013-03-22 18:22:00}
\pmmodified{2013-03-22 18:22:00}
\pmowner{pahio}{2872}
\pmmodifier{pahio}{2872}
\pmtitle{minimality property of Fourier coefficients}
\pmrecord{13}{41006}
\pmprivacy{1}
\pmauthor{pahio}{2872}
\pmtype{Theorem}
\pmcomment{trigger rebuild}
\pmclassification{msc}{42A16}
\pmclassification{msc}{42A10}
\pmrelated{CommonFourierSeries}
\pmrelated{UniquenessOfFourierExpansion}

% this is the default PlanetMath preamble.  as your knowledge
% of TeX increases, you will probably want to edit this, but
% it should be fine as is for beginners.

% almost certainly you want these
\usepackage{amssymb}
\usepackage{amsmath}
\usepackage{amsfonts}

% used for TeXing text within eps files
%\usepackage{psfrag}
% need this for including graphics (\includegraphics)
%\usepackage{graphicx}
% for neatly defining theorems and propositions
 \usepackage{amsthm}
% making logically defined graphics
%%%\usepackage{xypic}

% there are many more packages, add them here as you need them

% define commands here

\theoremstyle{definition}
\newtheorem*{thmplain}{Theorem}

\begin{document}
\PMlinkescapeword{polynomial}

Let $f$ be a Riemann integrable periodic real function with period $2\pi$ and $n$ a positive integer.\, Among all ``trigonometric polynomials''
$$\varphi(x) \,:=\, \frac{\alpha_0}{2}\!+\!\sum_{j=1}^n(\alpha_j\cos{jx}+\beta_j\sin{jx}),$$
the polynomial with the coefficients $\alpha_j$ and $\beta_j$ being the Fourier coefficients
$$\alpha_j = a_j := \frac{1}{\pi}\int_{-\pi}^{\pi} f(x)\cos{jx}\,dx$$
and
$$\beta_j = b_j := \frac{1}{\pi}\int_{-\pi}^{\pi} f(x)\sin{jx}\,dx$$
for the Fourier series of $f$ produces the minimal value of the mean square deviation
$$\frac{1}{2\pi}\int_{-\pi}^\pi[f(x)\!-\!\varphi(x)]^2\,dx.\\$$


{\em Proof.}\, For any fixed number $n$, it's a question of giving the least value to the definite integral
\begin{align}
m \;:=\; \frac{1}{2\pi}\int_{-\pi}^\pi\left[f(x)-\frac{\alpha_0}{2}\!-\!\sum_{j=1}^n(\alpha_j\cos{jx}+\beta_j\sin{jx})\right]^2dx 
\quad (\geqq 0).
\end{align}

Expanding $m$ and integrating termwise yields
\begin{align*}
m \;=\;\, & \frac{1}{2\pi}\int_{-\pi}^\pi(f(x))^2\,dx-\frac{\alpha_0}{2\pi}\int_{-\pi}^\pi f(x)\,dx\\
& -\frac{1}{\pi}\sum_{j=1}^n\alpha_j\int_{-\pi}^\pi f(x)\cos{jx}\,dx 
  -\frac{1}{\pi}\sum_{j=1}^n\beta_j \int_{-\pi}^\pi f(x)\sin{jx}\,dx
  +\frac{1}{2\pi}\frac{\alpha_0^2}{4}\int_{-\pi}^\pi dx\\
& +\frac{1}{2\pi}\sum_{j=1}^n\alpha_j^2\int_{-\pi}^\pi\cos^2{jx}\,dx 
  +\frac{1}{2\pi}\sum_{j=1}^n\beta_j^2 \int_{-\pi}^\pi\sin^2{jx}\,dx\\
& +\frac{\alpha_0}{2\pi}\sum_{j=1}^n\alpha_j\int_{-\pi}^\pi\cos{jx}\,dx 
  +\frac{\alpha_0}{2\pi}\sum_{j=1}^n\beta_j\,\int_{-\pi}^\pi\sin{jx}\,dx
+\frac{1}{\pi}\sum_{j=1}^n\sum_{k=1}^n\alpha_j\beta_k\!\int_{-\pi}^\pi\cos{jx}\,\sin{kx}\,dx\\
& +\frac{1}{\pi}\sum_{j=1}^n\sum_{k\neq j}\alpha_j\alpha_k\int_{-\pi}^\pi\cos{jx}\,\cos{kx}\,dx 
  +\frac{1}{\pi}\sum_{j=1}^n\sum_{k\neq j}\beta_j\beta_k\int_{-\pi}^\pi\sin{jx}\,\sin{kx}\,dx.
\end{align*}
Here, we have the Fourier coefficients
$$\frac{1}{\pi}\int_{-\pi}^\pi f(x)\,dx \;=\; a_0, \quad 
\frac{1}{\pi}\int_{-\pi}^\pi f(x)\cos{jx}\,dx \;=\; a_j, \quad \frac{1}{\pi}\int_{-\pi}^\pi f(x)\sin{jx}\,dx \;=\; b_j.$$
Furthermore,
$$\int_{-\pi}^\pi\cos^2{jx}\,dx = \int_{-\pi}^\pi\sin^2{jx}\,dx \;=\; \pi, \quad 
\int_{-\pi}^\pi\cos{jx}\,\sin{kx}\,dx \;=\; 0$$
and
$$\int_{-\pi}^\pi\cos{jx}\,\cos{kx}\,dx = \int_{-\pi}^\pi\sin{jx}\,\sin{kx}\,dx \;=\; 0 \quad \mbox{for\;\;} k \neq j.$$
Using all these we can write
$$m \;=\; \frac{1}{2\pi}\int_{-\pi}^\pi(f(x))^2\,dx-\frac{\alpha_0a_0}{2}-\sum_{i=1}^n(\alpha_ia_i+\beta_ib_i)+
\frac{a_0^2}{4}+\frac{1}{2}\sum_{i=1}^n(\alpha_i^2+\beta_i^2).$$
Adding and subtracting still the sum \,$\frac{a_0^2}{4}+\frac{1}{2}\sum_{i=1}^n(a_i^2+b_i^2)$\, yields finally the form
$$m \;=\; \frac{1}{2\pi}\int_{-\pi}^\pi(f(x))^2\,dx-\frac{a_0^2}{4}
-\frac{1}{2}\sum_{i=1}^n(a_i^2+b_i^2)+\frac{1}{4}(\alpha_0-a_0)^2
+\frac{1}{2}\sum_{i=1}^n[(\alpha_i-a_i)^2+(\beta_i-b_i)^2].$$
The three first addends of this sum do not depend on the choice of the quantities $\alpha_i$ and $\beta_i$.\, The other addends are non-negative, and their sum is minimal, equal 0, when 
$$\alpha_i = a_i, \quad \beta_i = b_i \quad \forall i.$$
Accordingly, the mean square deviation $m$, i.e. (1), is minimal when one uses the Fourier coefficients. Q.E.D.

\begin{thebibliography}{9}
\bibitem{NP}{\sc N. Piskunov:} {\em Diferentsiaal- ja integraalarvutus k\~{o}rgematele tehnilistele \~{o}ppeasutustele}.\, Kirjastus Valgus, Tallinn  (1966).
\end{thebibliography}

%%%%%
%%%%%
\end{document}
