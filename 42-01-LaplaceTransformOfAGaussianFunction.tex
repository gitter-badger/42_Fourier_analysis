\documentclass[12pt]{article}
\usepackage{pmmeta}
\pmcanonicalname{LaplaceTransformOfAGaussianFunction}
\pmcreated{2013-03-22 16:03:21}
\pmmodified{2013-03-22 16:03:21}
\pmowner{perucho}{2192}
\pmmodifier{perucho}{2192}
\pmtitle{Laplace transform of a Gaussian function}
\pmrecord{5}{38108}
\pmprivacy{1}
\pmauthor{perucho}{2192}
\pmtype{Application}
\pmcomment{trigger rebuild}
\pmclassification{msc}{42-01}

\endmetadata

% this is the default PlanetMath preamble.  as your knowledge
% of TeX increases, you will probably want to edit this, but
% it should be fine as is for beginners.

% almost certainly you want these
\usepackage{amssymb}
\usepackage{amsmath}
\usepackage{amsfonts}

% used for TeXing text within eps files
%\usepackage{psfrag}
% need this for including graphics (\includegraphics)
%\usepackage{graphicx}
% for neatly defining theorems and propositions
%\usepackage{amsthm}
% making logically defined graphics
%%%\usepackage{xypic}

% there are many more packages, add them here as you need them

% define commands here

\begin{document}
We evaluate the Laplace transform
{\footnote{cf. $\emph{Gaussian function}$, wikipedia.org}}
\begin{align}
\mathcal{L}\{e^{-t^2}\}=\int_0^\infty e^{-st}e^{-t^2}\,dt=F(s).
\end{align}
In fact,
\begin{align*}
\mathcal{L}\{e^{-t^2}\}=
\int_0^\infty e^{-(t^2+2\frac{s}{2}t+\frac{s^2}{4}-\frac{s^2}{4})}\,dt=
e^\frac{s^2}{4}\!\!\int_0^\infty e^{-(t+\frac{s}{2})^2}\,dt.
\end{align*}
By making the change of variable $t+\frac{s}{2}=u$, we have (by the second equality in (1), the variable on operator's argument is immaterial)
\begin{align*}
\mathcal{L}\{e^{-t^2}\}=
e^\frac{s^2}{4}\!\!\int_{\frac{s}{2}}^\infty e^{-u^2}\,du.
\end{align*}
That is,
\begin{align*}
\mathcal{L}\{e^{-t^2}\}=F(s)=
\frac{\sqrt{\pi}}{2}e^\frac{s^2}{4}\mathrm{erfc}\Big(\frac{s}{2}\Big),
\end{align*}
where $\mathrm{erfc}(\cdot)$ is the complementary error function. Its path of integration is subject to the restriction $\arg{(u)}\to\theta$, with
$|\theta|\leq\pi/4$ as $u\to\infty$ along the path, with equality only if $\Re{(u^2)}$ remains bounded to the left.

%%%%%
%%%%%
\end{document}
