\documentclass[12pt]{article}
\usepackage{pmmeta}
\pmcanonicalname{MicrolocalAnalysis}
\pmcreated{2013-03-22 16:35:25}
\pmmodified{2013-03-22 16:35:25}
\pmowner{PrimeFan}{13766}
\pmmodifier{PrimeFan}{13766}
\pmtitle{microlocal analysis}
\pmrecord{4}{38785}
\pmprivacy{1}
\pmauthor{PrimeFan}{13766}
\pmtype{Definition}
\pmcomment{trigger rebuild}
\pmclassification{msc}{42A38}

% this is the default PlanetMath preamble.  as your knowledge
% of TeX increases, you will probably want to edit this, but
% it should be fine as is for beginners.

% almost certainly you want these
\usepackage{amssymb}
\usepackage{amsmath}
\usepackage{amsfonts}

% used for TeXing text within eps files
%\usepackage{psfrag}
% need this for including graphics (\includegraphics)
%\usepackage{graphicx}
% for neatly defining theorems and propositions
%\usepackage{amsthm}
% making logically defined graphics
%%%\usepackage{xypic}

% there are many more packages, add them here as you need them

% define commands here

\begin{document}
In mathematical analysis, {\em microlocal analysis} is a term use to describe techniques developed from the 1950s onwards based on Fourier transforms related to the study of variable-coefficients-linear and nonlinear partial differential equations. This includes generalized functions, pseudo-differential operators, wave front sets, Fourier integral operators, and paradifferential operators.

The term microlocal implies localisation not just at a point, but in terms of cotangent space directions at a given point. This gains in importance on manifolds of dimension greater than one.

{\it This entry was adapted from the Wikipedia article \PMlinkexternal{Microlocal analysis}{http://en.wikipedia.org/wiki/Microlocal_analysis} as of January 17, 2007.}
%%%%%
%%%%%
\end{document}
