\documentclass[12pt]{article}
\usepackage{pmmeta}
\pmcanonicalname{PoissonSummationFormula}
\pmcreated{2013-03-22 13:27:25}
\pmmodified{2013-03-22 13:27:25}
\pmowner{rmilson}{146}
\pmmodifier{rmilson}{146}
\pmtitle{Poisson summation formula}
\pmrecord{16}{34022}
\pmprivacy{1}
\pmauthor{rmilson}{146}
\pmtype{Theorem}
\pmcomment{trigger rebuild}
\pmclassification{msc}{42A16}
\pmclassification{msc}{42A38}
\pmsynonym{Poisson summation}{PoissonSummationFormula}

\endmetadata

\usepackage{amsmath}
\usepackage{amsfonts}
\usepackage{amssymb}
\usepackage{amsthm}

\newcommand{\Rset}{\mathbb{R}}
\newcommand{\Zset}{\mathbb{Z}}
\newcommand{\hf}{\hat{f}}
\newcommand{\rLo}{L^{\!1}}
\newtheorem{theorem}{Theorem}
\begin{document}
Let $f:\Rset\to\Rset$ be an integrable function and let   
\[\hf(\xi)= \int_{\Rset} e^{-2\pi i\xi x} f(x)dx,\quad \xi\in\Rset.\] 
be its Fourier transform.   The Poisson summation formula is the assertion that
\begin{equation}
  \label{eq:psf}
  \sum_{n\in\Zset} f(n) = \sum_{n\in\Zset} \hf(n).
\end{equation}
whenever $f$ is such that both of the above infinite sums are
absolutely convergent.

Equation \eqref{eq:psf} is useful because it establishes a
correspondence between Fourier series and Fourier integrals.  To see
the connection, let
\[ g(x)=\sum_{n\in\Zset} f(x+n),\quad x\in\Rset, \] be the periodic
function obtained by pseudo-averaging\footnote{This terminology is at best a metaphor.  The operation in question is not a genuine mean, in the technical sense of that word.} $f$ relative to $\Zset$ acting
as the discrete group of translations on $\Rset$.  Since $f$ was
assumed to be integrable, $g$ is defined almost everywhere, and is
integrable over $[0,1]$ with
\[ \Vert g \Vert_{\rLo[0,1]}\leq \Vert f\Vert_{\rLo(\Rset)}.\]
Since $f$ is integrable, we may interchange integration and summation
to obtain
\[\hf(k) = \sum_{n\in\Zset} \int_0^1 f(x+n)e^{-2\pi ik x} dx =
\int_0^1 e^{-2\pi i k x} g(x) dx
\] 
for every $k\in\Zset$.  In other words, the restriction of the Fourier
transform of $f$ to the integers gives the Fourier coefficients of the
averaged, periodic function $g$.  Since we have assumed that the
$\hf(k)$ form an absolutely convergent series, we have that
\[ g(x) = \sum_{k\in\Zset} \hf(k) e^{2\pi ik x}\] in the sense of
uniform convergence.  Evaluating the above equation at $x=0$, we
obtain the Poisson summation formula \eqref{eq:psf}.
%%%%%
%%%%%
\end{document}
