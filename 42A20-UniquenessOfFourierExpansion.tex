\documentclass[12pt]{article}
\usepackage{pmmeta}
\pmcanonicalname{UniquenessOfFourierExpansion}
\pmcreated{2013-03-22 18:22:16}
\pmmodified{2013-03-22 18:22:16}
\pmowner{pahio}{2872}
\pmmodifier{pahio}{2872}
\pmtitle{uniqueness of Fourier expansion}
\pmrecord{5}{41012}
\pmprivacy{1}
\pmauthor{pahio}{2872}
\pmtype{Result}
\pmcomment{trigger rebuild}
\pmclassification{msc}{42A20}
\pmclassification{msc}{42A16}
\pmclassification{msc}{26A06}
\pmsynonym{uniqueness of Fourier series}{UniquenessOfFourierExpansion}
\pmrelated{FourierSineAndCosineSeries}
\pmrelated{MinimalityPropertyOfFourierCoefficients}
\pmrelated{DeterminationOfFourierCoefficients}
\pmrelated{ComplexSineAndCosine}
\pmrelated{UniquenessOfDigitalRepresentation}
\pmrelated{UniquenessOfLaurentExpansion}

\endmetadata

% this is the default PlanetMath preamble.  as your knowledge
% of TeX increases, you will probably want to edit this, but
% it should be fine as is for beginners.

% almost certainly you want these
\usepackage{amssymb}
\usepackage{amsmath}
\usepackage{amsfonts}

% used for TeXing text within eps files
%\usepackage{psfrag}
% need this for including graphics (\includegraphics)
%\usepackage{graphicx}
% for neatly defining theorems and propositions
 \usepackage{amsthm}
% making logically defined graphics
%%%\usepackage{xypic}

% there are many more packages, add them here as you need them

% define commands here

\theoremstyle{definition}
\newtheorem*{thmplain}{Theorem}

\begin{document}
If a real function $f$, Riemann integrable on the interval\, $[-\pi,\,\pi]$,\, may be expressed as sum of a trigonometric series
\begin{align}
f(x) = \frac{a_0}{2}\!+\!\sum_{m=1}^\infty(a_m\cos{mx}+b_m\sin{mx})
\end{align}
where the series $a_1\!+\!b_1\!+\!a_2\!+\!b_2\!+\!a_3\!+\!b_3\!+\ldots$ of the coefficients converges absolutely, then the series (1) converges uniformly on the interval and can be \PMlinkname{integrated termwise}{SumFunctionOfSeries}.\, The same concerns apparently the series which are gotten by multiplying the equation (1) by $\cos{nx}$ and by $\sin{nx}$;\, the results of the integrations determine for the coefficients $a_n$ and $b_n$ the unique values
\begin{align*}
a_n &= \frac{1}{\pi}\!\int_{-\pi}^{\pi} f(x)\cos{nx}\,dx,\\
b_n &= \frac{1}{\pi}\!\int_{-\pi}^{\pi} f(x)\sin{nx}\,dx
\end{align*}
for any $n$.\, So the Fourier series of $f$ is unique.\\

As a consequence, we can infer that the well-known goniometric formula
$$\sin^2{x} = \frac{1-\cos{2x}}{2}$$
presents the Fourier series \PMlinkescapetext{expansion} of the even function $\sin^2{x}$.

%%%%%
%%%%%
\end{document}
