\documentclass[12pt]{article}
\usepackage{pmmeta}
\pmcanonicalname{ExampleOfFourierSeries}
\pmcreated{2013-03-22 13:57:13}
\pmmodified{2013-03-22 13:57:13}
\pmowner{alozano}{2414}
\pmmodifier{alozano}{2414}
\pmtitle{example of Fourier series}
\pmrecord{10}{34718}
\pmprivacy{1}
\pmauthor{alozano}{2414}
\pmtype{Example}
\pmcomment{trigger rebuild}
\pmclassification{msc}{42A16}
\pmsynonym{example of Fourier coefficients}{ExampleOfFourierSeries}
\pmrelated{ValueOfTheRiemannZetaFunctionAtS2}
\pmrelated{FourierSineAndCosineSeries}

\endmetadata

% this is the default PlanetMath preamble.  as your knowledge
% of TeX increases, you will probably want to edit this, but
% it should be fine as is for beginners.

% almost certainly you want these
\usepackage{amssymb}
\usepackage{amsmath}
\usepackage{amsthm}
\usepackage{amsfonts}

% used for TeXing text within eps files
%\usepackage{psfrag}
% need this for including graphics (\includegraphics)
%\usepackage{graphicx}
% for neatly defining theorems and propositions
%\usepackage{amsthm}
% making logically defined graphics
%%%\usepackage{xypic}

% there are many more packages, add them here as you need them

% define commands here

\newtheorem{thm}{Theorem}
\newtheorem{defn}{Definition}
\newtheorem{prop}{Proposition}
\newtheorem{lemma}{Lemma}
\newtheorem{cor}{Corollary}

% Some sets
\newcommand{\Nats}{\mathbb{N}}
\newcommand{\Ints}{\mathbb{Z}}
\newcommand{\Reals}{\mathbb{R}}
\newcommand{\Complex}{\mathbb{C}}
\newcommand{\Rats}{\mathbb{Q}}
\begin{document}
Here we present an example of Fourier series:

{\bf Example:}

Let $f\colon (-\pi,\pi) \to \Reals$ be the ``identity'' function,
defined by
$$f(x)=x, \text{ for all }x\in (-\pi,\pi).$$
We will compute the Fourier coefficients for this function. Notice
that $\cos(nx)$ is an even function, while $f$ and $\sin(nx)$ are
odd functions.
\begin{eqnarray*}
a_0^f & =&
\frac{1}{2\pi}\int_{-\pi}^{\pi}f(x)dx=\frac{1}{2\pi}\int_{-\pi}^{\pi}
x dx= 0\\
a_n^f &=& \frac{1}{\pi}\int_{-\pi}^{\pi}f(x)\cos(nx)dx=
\frac{1}{\pi}\int_{-\pi}^{\pi}x \cos(nx)dx = 0\\
b_n^f &=&
\frac{1}{\pi}\int_{-\pi}^{\pi}f(x)\sin(nx)dx=\frac{1}{\pi}\int_{-\pi}^{\pi}
x \sin(nx)dx =\\
&=& \frac{2}{\pi}\int_{0}^{\pi} x\sin(nx) dx= \frac{2}{\pi}\left(
\left[-\frac{x\cos(nx)}{n}\right]_0^{\pi}+\left[\frac{\sin(nx)}{n^2}\right]_0^{\pi}=
\right)=(-1)^{n+1}\frac{2}{n}
\end{eqnarray*}

Notice that $a_0^f,a_n^f$ are $0$ because $x$ and $x \cos(nx)$ are
odd functions. Hence the Fourier series for $f(x)=x$ is:

\begin{eqnarray*}
 f(x)=x &=& a_0^f +
\sum_{n=1}^{\infty}(a_n^f\cos(nx)+b_n^f\sin(nx)) =\\
&=& \sum_{n=1}^{\infty}(-1)^{n+1}\frac{2}{n} \sin(nx), \quad \forall x\in (-\pi,\pi)
\end{eqnarray*}

For an application of this Fourier series, see value of the
Riemann zeta function at $s=2$.
%%%%%
%%%%%
\end{document}
