\documentclass[12pt]{article}
\usepackage{pmmeta}
\pmcanonicalname{ChristoffelDarbouxFormula}
\pmcreated{2013-03-22 16:20:32}
\pmmodified{2013-03-22 16:20:32}
\pmowner{rspuzio}{6075}
\pmmodifier{rspuzio}{6075}
\pmtitle{Christoffel-Darboux formula}
\pmrecord{9}{38474}
\pmprivacy{1}
\pmauthor{rspuzio}{6075}
\pmtype{Theorem}
\pmcomment{trigger rebuild}
\pmclassification{msc}{42C05}
\pmclassification{msc}{33D45}

% this is the default PlanetMath preamble.  as your knowledge
% of TeX increases, you will probably want to edit this, but
% it should be fine as is for beginners.

% almost certainly you want these
\usepackage{amssymb}
\usepackage{amsmath}
\usepackage{amsfonts}

% used for TeXing text within eps files
%\usepackage{psfrag}
% need this for including graphics (\includegraphics)
%\usepackage{graphicx}
% for neatly defining theorems and propositions
%\usepackage{amsthm}
% making logically defined graphics
%%%\usepackage{xypic}

% there are many more packages, add them here as you need them

% define commands here

\begin{document}
Let $\{\phi_i\}_{i=0}^n$ be orthonormal polynomials (the degree of $\phi_k$ is $k$) and let $k_n$ be the coefficient of $x^n$ in $\phi_n$.  Then
 \[\sum_{k=0}^n \phi_k (x) \phi_k (y) = {k_n \over k_{n+1}}
 \left({\phi_n (y) \phi_{n+1} (x) - \phi_n (x) \phi_{n+1} (y) \over
  x - y}\right)\]

The reason this formula is interesting is that the left-hand side is
the integral kernel for the projection operator to the subspace spanned
by the polynomials $\{\phi_i\}_{i=0}^n$.
%%%%%
%%%%%
\end{document}
